\chapter{绪论}
    \section{公式、表格与图片}
	
	\subsection{公式}
这是一个行内公式的例子:$1+1=2$。\par
这是一个公式的例子:
\begin{align*}
\mathbb{E}\left[\left(\frac{\sum_{i=1}^m X_i}{m} -\mathbb{E}[X]\right)^2\right]&=\mathbb{E}\left[\left(\frac{\sum_{i=1}^{m-1} X_i-(m-1)\mathbb{E}[X]}{m} +\frac{X_m-\mathbb{E}[X]}{m}\right)^2\right]\tag{1-1}\\
&=\frac{1}{m^2}\mathbb E\left[\left(\sum_{i=1}^{m-1} X_i-(m-1)\mathbb{E}[X]\right)^2\right]\label{1.2}\\
&\quad\ + \frac{1}{m^2} \mathbb{E}[(X_m-\mathbb{E}[X])^2]\\
&\quad\ +\frac{2}{m^2}\mathbb{E}\left[\left(\sum_{i=1}^{m-1} X_i-(m-1)\mathbb{E}[X]\right)(X_m-\mathbb{E}[X]) \right]\tag{1-2}\\
&=\frac{(m-1)^2}{m^2}\cdot\frac 1{m-1}(\mathbb{E}[X^2]-\mathbb{E}^2[X])\\
&\quad\ +\frac{1}{m^2}(\mathbb{E}[X^2]-\mathbb{E}^2[X])+0\tag{1-3}\\
&=\frac{\mathbb{E}[X^2]-\mathbb{E}^2[X]}{m}\tag{1-4}
\end{align*}

	\subsection{表格}
一个复杂表格的例子:
\begin{table*}[htb!]
\setlength{\abovecaptionskip}{0.05cm}  %段前
\setlength{\belowcaptionskip}{-0.1cm} %段后
\caption{$SGD$算法的收敛性能}
  \begin{center}
  \begin{threeparttable}
  \footnotesize
    \renewcommand{\TPTminimum}{\linewidth}
    \makebox[\linewidth]{%
    \tabcolsep=0.11cm
    \setlength{\extrarowheight}{2pt}
    \begin{tabular}{cccccc} 
\toprule[1.5pt]
\multirow{2}{*}{\bfseries{约束}\tnote{*}} & \multicolumn{2}{c}{\textbf{步长}} & \multirow{2}{*}{$\mathbb{E}[\epsilon(\tau)]$} & \multirow{2}{*}{$\mathbb{E}[\|\boldsymbol{\theta}^{(\tau)}-\boldsymbol{\theta}^* \|^2]$}&
\multirow{2}{*}{$\|\nabla \mathcal{L}(\boldsymbol{\theta}^{(\tau)})\|$}\\
\cline{2-3}& \multicolumn{1}{l}{\textbf{类型}} & 
\textbf{表达式}&&\\ \toprule[1.5pt]
\multirow{3.4}{*}{$L$-光滑\tnote{**}}& 固定&$\eta^{(\tau)}\equiv\eta>0$& - &- &$O(\frac 1{\eta\tau})+\frac{LG^2\eta}{2}$\\
& 衰减 & $\eta^{(\tau)}=\frac{c}{\tau+1}$ & -   & - & $O(\frac 1 {\log{\tau}})$   \\
& 衰减 & $\eta^{(\tau)}=\frac c{\sqrt{\tau+1}}$ & -   & - & $O(\frac{\log \tau}{\sqrt{\tau}})$ \\\hline

一般凸函数& 衰减 & $\eta^{(\tau)}=\frac c{\sqrt{\tau+1}}\in(0,\frac 1{2\mu})$ & $O(\frac 1{\sqrt{\tau}})$   & -& - \\
\bottomrule[1.5pt]
\end{tabular}}
\end{threeparttable}
\end{center}
\begin{tablenotes}
       \footnotesize
       \item{*} \zihao{-5}{这是表格注释1}
       \item{**} \zihao{-5}{这是表格注释2}
\end{tablenotes}
\end{table*}


    